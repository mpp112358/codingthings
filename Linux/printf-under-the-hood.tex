% Created 2025-03-06 Thu 23:59
% Intended LaTeX compiler: pdflatex
\documentclass[11pt]{article}
\usepackage[utf8]{inputenc}
\usepackage[T1]{fontenc}
\usepackage{graphicx}
\usepackage{longtable}
\usepackage{wrapfig}
\usepackage{rotating}
\usepackage[normalem]{ulem}
\usepackage{amsmath}
\usepackage{amssymb}
\usepackage{capt-of}
\usepackage{hyperref}
\author{Manuel Pérez}
\date{\today}
\title{Printf Under The Hood}
\hypersetup{
 pdfauthor={Manuel Pérez},
 pdftitle={Printf Under The Hood},
 pdfkeywords={},
 pdfsubject={},
 pdfcreator={Emacs 29.4 (Org mode 9.7.22)}, 
 pdflang={English}}
\begin{document}

\maketitle
\tableofcontents

\section{printf under the hood}
\label{sec:org06c327e}

\subsection{The program}
\label{sec:org73ef93c}

Will look into a really simple program that prints a text on a terminal and waits until a key is pressed:

\begin{verbatim}
sed -n 1~1p printfuth.cpp
\end{verbatim}

\begin{verbatim}
#include <bits/posix_opt.h>
#include <cstdio>
#include <termios.h>
#include <unistd.h>

int keypress() {
  struct termios saved_state, new_state;
  int c;
  if (tcgetattr(STDIN_FILENO, &saved_state) == -1) {
    return EOF;
  }
  new_state = saved_state;
  new_state.c_cflag &= ~(ICANON | ECHO);
  new_state.c_cc[VMIN] = 1;
  new_state.c_cc[VTIME] = 0;
  if (tcsetattr(STDIN_FILENO, TCSANOW, &new_state) == -1) {
    return EOF;
  }
  c = getchar();
  tcsetattr(STDIN_FILENO, TCSANOW, &saved_state);
  return c;
}

int main() {
  std::printf("Running on terminal: %s\n", ttyname(STDIN_FILENO));
  std::printf("Hello, world!");
  keypress();
  return 0;
}
\end{verbatim}
\subsection{The assembler instructions}
\label{sec:org11a8d25}

First things, first. The program is compiled into this (assembler):

\begin{verbatim}
g++ -g printfuth.cpp -o printfuth
objdump --disassemble=main printfuth > printfuth-disassemble
sed -n 1~1p printfuth-disassemble
\end{verbatim}

It can be seen that what the program does is basically a call to \texttt{printf}.
\subsubsection{{\bfseries\sffamily TODO} What's the purpose of \texttt{lea 0xeac(\%rip),\%rax}?}
\label{sec:org90e3094}

The LEA (Load Effective Address) instruction computes the effective address of a memory location and stores it in a register.

Here, it's storing the address where the string ``Hello, world!'' resides into the \texttt{rax} register. From that register it will be copied onto the \texttt{rdi} register, which is the one that must hold the first argument to a function called (according to the calling convention).

The address is expressed as an offset from the contents of the \texttt{rip} register (which is the instruction pointer). As it can be seen, the resulting address is \texttt{0x2004}, which equals \texttt{0x12dc + 0xd28} (the value of the \texttt{rip} register plus the offset \texttt{0xd28}).

The section of the executable file where the string is stored is this:

\begin{verbatim}
readelf -x .rodata printfuth
\end{verbatim}
\subsubsection{{\bfseries\sffamily TODO} How is the argument passed to \texttt{printf}?}
\label{sec:org176325c}
\subsection{{\bfseries\sffamily TODO} Execution context}
\label{sec:orga830c9c}

The program is run by a shell, whose standard output is directed to a terminal window (a terminal emulator window, to be precise) on the screen.
\subsection{{\bfseries\sffamily TODO} The system calls}
\label{sec:org2057da0}

\begin{verbatim}
strace ./printfuth 2>&1 1>/dev/null
\end{verbatim}

The key system call here is \texttt{write(1, "Hello, world!", 13)}.

It writes to the file with descriptor 1, which is, by default, the standard output in Linux.
\subsection{Scheme}
\label{sec:org51dbd2b}

Looks like this:
\begin{itemize}
\item \texttt{printf} makes a \texttt{write} syscall.
\item \texttt{write} syscall writes the string to file with file descriptor 1, which is, by default, the \emph{standard output}.
\item The file descriptor 1 (\texttt{STDOUT}) is handled by a driver that controls a \emph{device} which is a pseudoterminal (something like \texttt{/dev/pty/3}).
\item The pseudoterminal is connected to a terminal emulator aplication (the one we are running the command from).
\item The terminal emulator receives the characters and prints them onto its window.
\end{itemize}

Still obscure parts:
\begin{itemize}
\item What does it mean that the file descriptor 1 is handled by a driver that controls a pseudoterminal device?
\item How does the terminal emulator write characters on the window (how are shapes, sizes, etc., handled)?
\item What are the details about how the pseudoterminal is connected to the terminal emulator?
\item What processes are forked, exec'ed or the like for all these things to happen?
\end{itemize}
\end{document}
